\documentclass[oneside,10pt]{article}
\usepackage[latin1]{inputenc}
\usepackage[francais]{babel}
\usepackage[francais]{layout}
\usepackage[OT1]{fontenc}
\usepackage{listings}
\usepackage{cite}
\usepackage{textcomp}
\usepackage{graphicx}
\usepackage[official]{eurosym}


% Reglages du document
\lstset{language=bash, frame=single, breaklines=true, basicstyle=\ttfamily, keywordstyle=\bfseries}
\setlength{\hoffset}{-18pt}        
\setlength{\oddsidemargin}{0pt} % Marge gauche sur pages impaires
\setlength{\evensidemargin}{9pt} % Marge gauche sur pages paires
\setlength{\marginparwidth}{54pt} % Largeur de note dans la marge
\setlength{\textwidth}{481pt} % Largeur de la zone de texte (17cm)
\setlength{\voffset}{-18pt} % Bon pour DOS
\setlength{\marginparsep}{7pt} % Séparation de la marge
\setlength{\topmargin}{0pt} % Pas de marge en haut
\setlength{\headheight}{13pt} % Haut de page
\setlength{\headsep}{10pt} % Entre le haut de page et le texte
\setlength{\footskip}{27pt} % Bas de page + séparation
\setlength{\textheight}{708pt} % Hauteur de la zone de texte (25cm)

\begin{document}

% Page de couverture
\title{Proposition d'architecture r\'eseau}
\author{Louis BILLIET \\ Florent DAVID}
\date{17 Janvier 2013}
\maketitle

\section{Cahier des charges}
l'architecture \`a proposer doit r\'epondre au contrainte suivante:
\begin{itemize}
  \item 2 b\^atiments de 500m$^{2}$
  \item Une salle de conf\'erence est pr\'evue avec un acc\`es invit\'e permettant d'aller sur internet uniquement
  \item Un r\'eseau local avec un acc\`es sur un NAS partag\'e de vous devrez acheter et administrer
  \item Assurer une connectivit\'e \`a 5 Mbps par prise jusqu'\`a ce NAS et dans le r\'eseau
  \item Un acc\`es internet redond\'e et s\'ecuris\'e
  \item Un r\'eseau de management de vos \'equipements r\'eseaux
  \item Une infrastructure robuste qui g\`ere les cas de pannes jusqu'\`a 10 utilisateurs en panne maximum
  \item Une infrastructure \'evolutive
  \item Les \'equipements seront isol\'es des employ\'es dans un local technique situ\'e dans un local technique par \'etage
  \item Les bureaux sont dispos\'es comme suit:
  \begin{itemize}
    \item 1 \'etage: 15 bureaux de 2 personnes + une salle de conf\'erence + 1 imprimante
    \item 2eme: 2 bureaux OpenSpace de 15 personnes + 2 imprimantes
    \item 3eme \'etage: Bureaux de la direction, 1 bureau + 1 imprimante par personnes 
  \end{itemize}
\end{itemize}

\section{Sch\'emas}
Inclure plein de dessins compliqu\'es ici !

\section{Co\^uts}
\begin{tabular}{|c|c|c|c|c|}
  \hline
  D\'esignation & r\'ef\'erence & Co\^ut unitaire & Nombre & Co\^ut total \\
  \hline \hline
  Switch 8 ports & Netgear GS308 & 28.90 \euro & 34 & 982.60 \euro \\
  \hline
  Routeur 4 ports & Cisco Small Business RV180 & 116.90 \euro & 2 & 233.80 \euro \\
  \hline
  Switch param\'etrable 8 ports & D-Link DGS-1100-08 & 59.90 \euro & 22 & 1317.8 \euro \\
  \hline
  Point d'acc\`es Wi-fi & TP-LINK TL-WA701ND & 29.90 \euro & 2 & 59.8 \euro \\
  \hline
  NAS & Synology RackStation RS814 & 579.95 \euro & 1 & 579.95 \euro \\
  \hline
  C\^able r\'eseau & & 1000 & ??? & 1000 \\
  \hline
  Prise RJ45 m\^ale & & 1000 & ??? & 1000 \\
  \hline
  Prise murale RJ45 & & 1000 & ??? & 1000 \\
  \hline
\end{tabular}

\end{document}
