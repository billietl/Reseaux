\documentclass[oneside,10pt]{article}
\usepackage[latin1]{inputenc}
\usepackage[francais]{babel}
\usepackage[francais]{layout}
\usepackage[OT1]{fontenc}
\usepackage{listings}
\usepackage{cite}
\usepackage{textcomp}

% Réglages du document
\lstset{language=bash, frame=single, breaklines=true, basicstyle=\ttfamily, keywordstyle=\bfseries}
\setlength{\hoffset}{-18pt}        
\setlength{\oddsidemargin}{0pt} % Marge gauche sur pages impaires
\setlength{\evensidemargin}{9pt} % Marge gauche sur pages paires
\setlength{\marginparwidth}{54pt} % Largeur de note dans la marge
\setlength{\textwidth}{481pt} % Largeur de la zone de texte (17cm)
\setlength{\voffset}{-18pt} % Bon pour DOS
\setlength{\marginparsep}{7pt} % Séparation de la marge
\setlength{\topmargin}{0pt} % Pas de marge en haut
\setlength{\headheight}{13pt} % Haut de page
\setlength{\headsep}{10pt} % Entre le haut de page et le texte
\setlength{\footskip}{27pt} % Bas de page + séparation
\setlength{\textheight}{708pt} % Hauteur de la zone de texte (25cm)

\begin{document}

% Page de couverture
\title{Proposition de solution : tunnel HTTP}
\author{Louis BILLIET \\ Florent DAVID}
\date{12 Nov. 2013}
\maketitle

\section{Mode de fonctionnemnt}
\subsection{Le script derri\`ere le proxy : work.py}
Le script demande \`a l'utilisateur quel service (port) rendre disponible depuis l'autre bout du tunnel.
Il cr\'eera ensuite un thread afin d'assurer la communication entre le service rendu disponible et le script \verb+home.py+.
Ce thread utilisera deux buffers afin de temporiser la discution avec le service.
Le thread principal communiquera ensuite avec le script \verb+home.py+ en HTTP via des requ\^etes POST, en utilisant les m�mes buffers que le second thread pour faire transiter les donn�es.
Losqu'on re�oit un code de r\'eponse 410, on vide les buffers et on renouvelle la connection avec le service local, pour �muler la connection d'un nouveau client.

\subsection{Le script sur l'internet : home.py}
Le script ouvre une socket afin d'accueillir la connection du client qui va traverser le tunnel, et en communiquera le port \`a l'utilisateur.
Un thread sera cr\'e\'e comme dans le script \verb+work.py+ pour assurer la communication avec le client local.
Le thread principal lancera un serveur HTTP customis\'e qui receptionnera les requ\^etes du script \verb+work.py+.
Lorsque le client ferme la connection, le serveur HTTP envoie un code de r�ponse 410 et ferme toute connection pour terminer.

\section{Les stat\'egies de d\'etection contourn\'es}
\subsection{Contournement de la politique de cache}
Dans chaque requ\^ete, le header ``Cache-Control'' pr\'ecise au serveur de ne pas cacher la r�ponse.
De m\^eme pour les r\'eponses.
Ainsi, un proxy normalement configur\'e respectant les normes sera oblig\'e de faire suivres toutes les requ\^etes vers le serveur.

\subsection{Pr\'ecision du type de donn\'e envoy\'es}
Les requ\^etes pr�cisent que \verb+work.py+ acc\`epte l'encodage en base64 uniquement.
Encodage qu'utilise \verb+home.py+ pour transf\'erer les donn\'ees.

\subsection{Vous avez dit ``port exotique'' ?}
Le serveur web du script \verb+home.py+ \'ecoute sur le port 80.
Les proxy restreignant les ports accessibles ne peuvent bloquer le port 80.
A moins de bloquer 90\% du trafic de l'internet.

\subsection{Trop de requ\^ete vers la m\^eme ressource ?}
Chaque requ\^ete concerne un fichier html dont le nom fait 20 caract\`eres de long.
Il y a donc peu de chances de consulter 2 fois la m\^eme ressource en 1 seconde.

\subsection{La requ\^ete de taille tr\`es correcte}
Le corps de la requ\^ete fait 0 octet, \'etant donn\'e qu'elle est vide.
La taille annonc\'e est donc de 0, pour les proxies un peu trop curieux.

\end{document}
