\documentclass[oneside,10pt]{article}
\usepackage[latin1]{inputenc}
\usepackage[francais]{babel}
\usepackage[francais]{layout}
\usepackage[OT1]{fontenc}
\usepackage{listings}
\usepackage{cite}
\usepackage{textcomp}
\usepackage{hyperref}

% Réglages du document
\lstset{language=bash, frame=single, breaklines=true, basicstyle=\ttfamily, keywordstyle=\bfseries}
\setlength{\hoffset}{-18pt}        
\setlength{\oddsidemargin}{0pt} % Marge gauche sur pages impaires
\setlength{\evensidemargin}{9pt} % Marge gauche sur pages paires
\setlength{\marginparwidth}{54pt} % Largeur de note dans la marge
\setlength{\textwidth}{481pt} % Largeur de la zone de texte (17cm)
\setlength{\voffset}{-18pt} % Bon pour DOS
\setlength{\marginparsep}{7pt} % Séparation de la marge
\setlength{\topmargin}{0pt} % Pas de marge en haut
\setlength{\headheight}{13pt} % Haut de page
\setlength{\headsep}{10pt} % Entre le haut de page et le texte
\setlength{\footskip}{27pt} % Bas de page + séparation
\setlength{\textheight}{708pt} % Hauteur de la zone de texte (25cm)

\begin{document}

% Page de couverture
\title{Proposition de solution : proxy HTTP}
\author{Louis BILLIET \\ Alexandre LOYWICK}
\date{12 Nov. 2013}
\maketitle

\section{Mode de fonctionnement}
Notre proxy est bas� sur \href{http://www.decalage.info/python/cherryproxy}{Cherryproxy}, un proxy HTTP extensible \'ecrit en python2.
Cherryproxy est bas\'e sur \href{http://docs.cherrypy.org/dev/refman/wsgiserver/init.html}{Cherrypy WSGI} et httplib et est diffus\'e sous licence BSD.
Il est annonc\'e comme ne supportant pas le protocole HTTPS (ce qui n'est pas grave dans le cadre su TP).
Ce qui est d\'erangeant par contre, c'est qu'il ne supporte pas toutes les fonctionnalit\'es de HTTP 1.1 : la m\'ethode POST, par exemple, est mal support\'e.
Ce qui nous a cout\'e pas mal de d\'efaites lors du deathmatch car le proxy est jug\'e ``trop s\'ev\`ere''.\\
\\
Pour s'en servir, il suffit d'h\'eriter d'une classe (cherryproxy.CherryProxy) et de red\'efinir 3 m\'ethodes :
\begin{itemize}
\item \verb+filter_request_headers+, qui sera appel\'e lorsqu'on re\c coit une requ\^ete.
\item \verb+filter_request+, qui sera appel\'e lorsque le corps de la requ\^ete est lue (s'il y en a une).
\item \verb+filter_response+, qui sera appel\'e lorsqu'on r\'ecup\`ere la r\'eponse d'une requ\^ete non filtr\'ee.
\end{itemize}

\section{Les strat�gies de d�tection mises en place}
Ce chapitre traitera des diff\'erents message laiss\'es par notre proxy.

\subsection{Ai-je bien lu 'SSH' ?}
Lorsqu'on ouvre une connexion vers un serveur SSH avec telnet, on re�oit un prompt ressemblant \`a \verb+SSH-2.0-OpenSSH_6.4+.
Si on filtre ``OpenSSH\_'' (qui a tr\`es peu de chances d'appara\^itre dans une requ\^ete l\'egitime), et ce dans tous les codages possibles (base 64, base 32, binascii, ...), le client SSH ne recevra pas la prompt et la communication ne pourra pas s'initialiser.

\subsection{User-Agent vide ou incorrect !}
Tout bon logiciel \'emanant des requ\^etes HTTP utilisent un user-agent.
L'absence de ce header est possible mais extr\^emement rare.

\subsection{Je suis un proxy web ! Tu m'entends ? WEB !}
Un filtrage simple : si l'utilisateur consulte un service sur un port autre que 80 (HTTP) et 443 (HTTPS), la connexion est refus\'ee.

\subsection{Tu sais quoi ? Ton serveur t'as repondu de la merde !}
On regarde s'il y a correspondance entre le \verb+accept-encoding+ de la requ\^ete et le \verb+content-encoding+ de la r\'eponse.
S'il n'y a pas correspondance, c'est soit une tentative d'attaque (!), soit un logiciel qui tente de faire des trucs pas nets.

\section{Conclusion}
Ne pas utiliser Cherryproxy. Ne pas attendre non plus une mise \`a jour car le projet \`a l'air abandonn\'e depuis Novembre 2011\ldots

\end{document}
