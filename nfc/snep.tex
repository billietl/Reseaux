\documentclass[oneside,10pt]{article}
\usepackage[latin1]{inputenc}
\usepackage[francais]{babel}
\usepackage[francais]{layout}
\usepackage[OT1]{fontenc}
\usepackage{listings}
\usepackage{cite}
\usepackage{url}
\usepackage{textcomp}

% Réglages du document
\lstset{language=bash, frame=single, breaklines=true, basicstyle=\ttfamily, keywordstyle=\bfseries}
\setlength{\hoffset}{-18pt}        
\setlength{\oddsidemargin}{0pt} % Marge gauche sur pages impaires
\setlength{\evensidemargin}{9pt} % Marge gauche sur pages paires
\setlength{\marginparwidth}{54pt} % Largeur de note dans la marge
\setlength{\textwidth}{481pt} % Largeur de la zone de texte (17cm)
\setlength{\voffset}{-18pt} % Bon pour DOS
\setlength{\marginparsep}{7pt} % S�paration de la marge
\setlength{\topmargin}{0pt} % Pas de marge en haut
\setlength{\headheight}{13pt} % Haut de page
\setlength{\headsep}{10pt} % Entre le haut de page et le texte
\setlength{\footskip}{27pt} % Bas de page + s�paration
\setlength{\textheight}{708pt} % Hauteur de la zone de texte (25cm)

\begin{document}

% Page de couverture
\title{Documentation : Simple NDEF Exchange Protocol}
\author{Louis BILLIET \\ Florent DAVID}
\date{23 Oct. 2013}
\maketitle

\section{Introduction}

Le Simple NDEF Exchange Protocol (SNEP) permet \`a une application s'executant sur un appareil \'equip\'e pour le NFC d'\'echanger des messages NDEF (NFC Data Exchange Format) avec un autre appareil \'equip\'e lors d'une communication en mode pair \`a pair. Le protocole se base sur le protocole Logical Link Control Protocol (LLCP) afin d'\'etablir une liaison fiable lors de l'\'echange de donn\'ees.\cite{NFC_forum}


Ceci est une mauvaise r\'ef\'erence \cite{Wikipedia_NFC}.
\section{Un autre chapitre}

\section{Bibliographie}

% Bibliographie
\bibliographystyle{unsrt}
\bibliography{biblio}{}

\end{document}
